\documentclass[12pt,twoside]{article}
\usepackage{jmlda}
%\NOREVIEWERNOTES
\title
    [Распознавание текста на основе скелетного представления толстых линий и сверточных сетей] 
    {Распознавание текста на основе скелетного представления толстых линий и сверточных сетей}
\author
    [Григорьев А.\,Д., Местецкий Л.\,М., Райер И.\, А, Жариков И.\,Н, Стрижов В.\,В.] 
    {Григорьев А.\,Д., Местецкий Л.\,М., Райер И.\, А, Жариков И.\,Н, Стрижов В.\,В.}
    [Григорьев А.\,Д., Местецкий Л.\,М., Райер И.\, А, Жариков И.\,Н, Стрижов~В.\,В.] 
\thanks
    {Научный руководитель:  Стрижов~В.\,В.
    Задачу поставил:  Местецкий Л.\, М., Райер И.\, А
    Консультант:  Жариков И.\,Н.}
\email
    {grigorev.ad@phystech.edu; mestlm@mail.ru; reyer@forecsys.ru; zharikov.i.n@yandex.ru; strijov@phystech.edu}
\organization
    {Московский физико-технический институт}
\abstract
    {В работе рассматривается задача распознования символов на изображении. Предлагается развить альтернативный способ решения, основанный на медиальном представлении фигуры. С целью улучшения качества распознования производится построение сверточной нейронной сети, получающей на вход граф скелетного представления объекта. Оценка качества предложенного метода проводится на корпусе  \textcolor{purple}{изображений арабских цифр}. Полученный алгоритм сравнивается с базовым -- сверточной нейронной сетью, работающей с изображением в растровом представлении.

\bigskip
\textbf{Ключевые слова}: \emph {распознавание символов, сверточная нейронная сеть, медиальное представление, скелетонизация}.}
\titleEng
    {JMLDA paper example: file jmlda-example.tex}
\authorEng
    {Author~F.\,S.$^1$, CoAuthor~F.\,S.$^2$, Name~F.\,S.$^2$}
\organizationEng
    {$^1$Organization; $^2$Organization}
\abstractEng
    {This document is an example of paper prepared with \LaTeXe\
    typesetting system and style file \texttt{jmlda.sty}.

    \bigskip
    \textbf{Keywords}: \emph{keyword, keyword, more keywords}.}
\begin{document}
\maketitle
%\linenumbers
\section{Введение}
В работе решается задача распознавания символов на изображении. Данная проблема возникает при распознавании текста после сегментации на символы, что имеет множество применений на практике, например, при распознавании содержимого документов на фотографии или отцифровке старых книг.

Стандартные существующие методы решения этой задачи предполагают работу с дискретным представлением изображения в виде матрицы пикселей, каждый элемент которой несет информацию о цвете пикселя в формате RGB.  Такой формат представления изображений называется растровым. Данный тип графики может быть быстро и эффективно обработан компьютером, в связи с чем широко распространен. Работе с растровыми изображениями посвящено большое число работ по классификации объектов. В последнее время с появлением нейронных сетей качество такой классификации существенно возросло. В частности сверточные нейронные сети в данной задаче позволяют добиться достаточно высокой точности, так ансамбль сверточных нейронных сетей в задаче классификации  рукописных цифр набора данных MNIST дает точность $99.77\%$ \cite{conf/icdar/CiresanMGS11}.

В данной работе предлагается развить другой подход работы с изображениями, основанный на непрерывном представлении в виде скелета. Основы данного подхода, а также алгоритмы скелетонизации описаны в работах Местецкого Л.~M.\cite{mest2009, mest2006, mest2008}. Скелетное представление фигуры может быть представлено в виде графа, алгоритмы его построения были описаны и реализованы \cite{journals/corr/Fujita15, journals/corr/DirnbergerNK15, skan_lib}. Ранее были предприняты попытки построения классификаторов на основе графового представления скелета фигуры, использовались такие модели, как SVM и случайный лес \cite{Svm_cush, rf_msu}. Предполагается, что построение графовой сверточной нейронной сети позволит улучшить качество классификации, достигнутое в предыдущих исследованиях.

Существующие сверточные нейронные сети, принимающие на вход граф, можно разделить на два класса: основанные на спектральном и пространственном подходе соответственно. В данной работе рассматриваются пространственные методы, поскольку граф, сгенерированный по изображению, имеет произвольный вид. \textcolor{purple}{Предлагается развить подход пространственой фильтрации, описанный в работе \cite{rsf}, адаптировав его к решению рассматриваемой задачи}.



\section{Название раздела}
TODO

\paragraph{Название параграфа.}
TODO

\section{Заключение}
TODO

\bibliographystyle{plain}
\bibliography{biblio}
%\begin{thebibliography}{1}

%\bibitem{author09anyscience}
%    \BibAuthor{Author\;N.}
%    \BibTitle{Paper title}~//
%    \BibJournal{10-th Int'l. Conf. on Anyscience}, 2009.  Vol.\,11, No.\,1.  Pp.\,111--122.
%\bibitem{myHandbook}
%    \BibAuthor{Автор\;И.\,О.}
%    Название книги.
%    Город: Издательство, 2009. 314~с.
%\bibitem{author09first-word-of-the-title}
%    \BibAuthor{Автор\;И.\,О.}
%    \BibTitle{Название статьи}~//
%    \BibJournal{Название конференции или сборника},
%    Город:~Изд-во, 2009.  С.\,5--6.
%\bibitem{author-and-co2007}
%    \BibAuthor{Автор\;И.\,О., Соавтор\;И.\,О.}
%    \BibTitle{Название статьи}~//
%    \BibJournal{Название журнала}. 2007. Т.\,38, \No\,5. С.\,54--62.
%\bibitem{bibUsefulUrl}
%    \BibUrl{www.site.ru}~---
%    Название сайта.  2007.
%\bibitem{voron06latex}
%    \BibAuthor{Воронцов~К.\,В.}
%    \LaTeXe\ в~примерах.
%    2006.
%    \BibUrl{http://www.ccas.ru/voron/latex.html}.
%\bibitem{Lvovsky03}
%    \BibAuthor{Львовский~С.\,М.} Набор и вёрстка в пакете~\LaTeX.
%    3-е издание.
%    Москва:~МЦHМО, 2003.  448~с.
%\end{thebibliography}

% Решение Программного Комитета:
%\ACCEPTNOTE
%\AMENDNOTE
%\REJECTNOTE
\end{document}
