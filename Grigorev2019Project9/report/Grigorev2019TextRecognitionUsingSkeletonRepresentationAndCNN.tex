\documentclass[12pt,twoside]{article}
\usepackage{jmlda}
%\NOREVIEWERNOTES
\title
    [Распознавание текста на основе скелетного представления толстых линий и сверточных сетей] 
    {Распознавание текста на основе скелетного представления толстых линий и сверточных сетей}
\author
    [Григорьев А.\,Д., Местецкий Л.\,М., Райер И.\, А, Жариков И.\,Н, Стрижов В.\,В.] 
    {Григорьев А.\,Д., Местецкий Л.\,М., Райер И.\, А, Жариков И.\,Н, Стрижов В.\,В.}
    [Григорьев А.\,Д., Местецкий Л.\,М., Райер И.\, А, Жариков И.\,Н, Стрижов~В.\,В.] 
\thanks
    {Научный руководитель:  Стрижов~В.\,В.
    Задачу поставил:  Местецкий Л.\, М., Райер И.\, А
    Консультант:  Жариков И.\,Н.}
\email
    {grigorev.ad@phystech.edu; mestlm@mail.ru; reyer@forecsys.ru; zharikov.i.n@yandex.ru; strijov@phystech.edu}
\organization
    {Московский физико-технический институт}
\abstract
    {В работе рассматривается задача распознования символов на изображении. Предлагается новый способ решения, основанный на медиальном представлении фигуры. Производится построение сверточной нейронной сети, получающей на вход признаковое описание графа скелетного представления объекта. Оценка качества предложенного метода проводится на корпусе  \textcolor{purple}{изображений арабских цифр в растровом представлении}.

\bigskip
\textbf{Ключевые слова}: \emph {распознавание символов, сверточная нейронная сеть, медиальное представление изображений, скелетонизация, векторное представление графов}.}
\titleEng
    {JMLDA paper example: file jmlda-example.tex}
\authorEng
    {Author~F.\,S.$^1$, CoAuthor~F.\,S.$^2$, Name~F.\,S.$^2$}
\organizationEng
    {$^1$Organization; $^2$Organization}
\abstractEng
    {This document is an example of paper prepared with \LaTeXe\
    typesetting system and style file \texttt{jmlda.sty}.

    \bigskip
    \textbf{Keywords}: \emph{keyword, keyword, more keywords}.}
\begin{document}
\maketitle
%\linenumbers
\section{Введение}
TODO

\section{Название раздела}
TODO

\paragraph{Название параграфа.}
TODO

\section{Заключение}
TODO


\bibliographystyle{unsrt}
\bibliography{jmlda-bib}
%\begin{thebibliography}{1}

%\bibitem{author09anyscience}
%    \BibAuthor{Author\;N.}
%    \BibTitle{Paper title}~//
%    \BibJournal{10-th Int'l. Conf. on Anyscience}, 2009.  Vol.\,11, No.\,1.  Pp.\,111--122.
%\bibitem{myHandbook}
%    \BibAuthor{Автор\;И.\,О.}
%    Название книги.
%    Город: Издательство, 2009. 314~с.
%\bibitem{author09first-word-of-the-title}
%    \BibAuthor{Автор\;И.\,О.}
%    \BibTitle{Название статьи}~//
%    \BibJournal{Название конференции или сборника},
%    Город:~Изд-во, 2009.  С.\,5--6.
%\bibitem{author-and-co2007}
%    \BibAuthor{Автор\;И.\,О., Соавтор\;И.\,О.}
%    \BibTitle{Название статьи}~//
%    \BibJournal{Название журнала}. 2007. Т.\,38, \No\,5. С.\,54--62.
%\bibitem{bibUsefulUrl}
%    \BibUrl{www.site.ru}~---
%    Название сайта.  2007.
%\bibitem{voron06latex}
%    \BibAuthor{Воронцов~К.\,В.}
%    \LaTeXe\ в~примерах.
%    2006.
%    \BibUrl{http://www.ccas.ru/voron/latex.html}.
%\bibitem{Lvovsky03}
%    \BibAuthor{Львовский~С.\,М.} Набор и вёрстка в пакете~\LaTeX.
%    3-е издание.
%    Москва:~МЦHМО, 2003.  448~с.
%\end{thebibliography}

% Решение Программного Комитета:
%\ACCEPTNOTE
%\AMENDNOTE
%\REJECTNOTE
\end{document}
